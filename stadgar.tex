\documentclass[11pt,a4paper]{article}
%% Setup
\usepackage[utf8]{inputenc}
\usepackage{amsmath}
\usepackage{graphicx}
\usepackage{amsfonts}
\usepackage{amssymb}
\usepackage{caption}
\usepackage{hyperref}
\usepackage{float}
\usepackage{tabularx}
\usepackage{fancyhdr}
\usepackage[swedish]{babel}  
\usepackage[left=2cm,right=2cm,top=2cm,bottom=2.5cm]{geometry}
\usepackage{lastpage}
\usepackage{tabularx}

\usepackage{listings}
\lstset{literate=%
{ö}{{\o}}1
{ä}{{\ae}}1
{æ}{{\ae}}1
{å}{{\aa}}1
{ø}{{\o}}1
{Æ}{{\AE}}1
{Å}{{\AA}}1
{Ø}{{\O}}1
}

\lstset{extendedchars=\true}
\lstset{inputencoding=ansinew}

\renewcommand{\thesection}{\S\arabic{section}}

\title{\textbf{Stadgar}}
\date{Antagna 30 september 2013}
\author{\textbf{LiTHe Kod}}

\begin{document}
\maketitle
\thispagestyle{empty}
\pagebreak
\tableofcontents
\pagebreak

\section*{Ändringshistorik}
% X i en tabularx gör att kolumnen expanderar i sidled så mycket som möjligt.
\begin{tabularx}{\textwidth}{X l l}
    \bfseries Motion/proposition & \bfseries Diskuterad & \bfseries Införd \\\hline
    \vspace{-0.6em} % Ä:et nedan gick in i linjen ovanför annars.
    Införande av ändringshistorik i stadgarna & 2022--05--10 & 2022--05--10 \\
    Ändring av stadgarna för att efterleva GDPR & 2022--05--10 & 2022--05--10 \\
    Förtydligande av hållpunkter inför stormöten & 2022--05--10 & 2022--05--10 \\
\end{tabularx}
\pagebreak

\pagestyle{fancy}
\lhead{\includegraphics[width=30mm]{lithekod.png}}
\rhead{Stadgar för LiTHe Kod \\ Sida \thepage\ av \pageref{LastPage}}
\lfoot{Linköping 30 september 2013}
\rfoot{www.lithekod.se}

\section{Föreningen}
\subsection{Namn}
Förenings namn är LiTHe Kod. Nedan kallad ``Föreningen''.
\subsection{Säte}
Föreningens säte är Linköpings universitet.
\subsection{Syfte}\label{sec:syfte}
Föreningens syfte är att främja och inspirera intresset för programmering, mjukvaruutveckling och andra relaterade ämnen.
\subsection{Oberoende och organisation}
Föreningen drivs ideellt, är partipolitiskt och religiöst oberoende sammanslutning. Föreningens åsikter och ställningstaganden är inte heller representativa för föreningens samarbetspartners eller vise versa.
\subsection{Upplösning}
För upplösning av föreningen krävs godkännande av minst 90\% majoritet av de röstande på två på varandra följande stormöten. Vid upplösning ska, efter att eventuella skulder betalats, kvarvarande likvida medel tillfalla en ideell organisation vald av föreningens medlemmar.

\section{Medlemsskap}\label{sec:member}
\subsection{Studerandemedlem}
En person äger rätt till studerandemedlemskap så länge han eller hon är registrerad som student vid Linköping universitet.
\subsection{Stödmedlem}
En person som inte har rätt till studerandemedlemskap kan bli stödmedlem och därigenom tillåtas deltaga i föreningens aktiviteter.

\subsection{Medlemsavgift}
Medlemskap erhålls för ett verksamhetsår genom inskrivning i medlemsregistret och betalning av fastställd medlemsavgift.
\subsubsection{Inaktivt medlemskap}
Medlemskap inaktiveras om medlemsavgift ej betalats inom två (2) månader
efter påbörjat verksamhetsår. Om medlemskap är gratis inaktiveras medlemskap om styrelsen inte kontaktats inom två (2) månader efter påbörjat verksamhetsår. Inaktiva medlemmar har inga rättigheter i föreningen.
\subsubsection{Undantag}
Sittande styrelsemedlemmar samt revisorn betalar ingen medlemsavgift men är ändå medlemmar i föreningen.
\subsection{Uppsägande av medlemskap}
En medlem kan begära utgång ur föreningen genom att skriftligt meddela det till styrelsen som sedan skall behandla ärendet skyndsamt.
\subsection{Uteslutande av medlem}
Styrelsen äger rätt att utesluta medlem från föreningen fram till nästa
stormöte om denne har motarbetat förenings arbete eller skadat föreningens verksamhet eller anseende. Den uteslutna måste få möjlighet att försvara sig inför styrelsen före beslutet. Utesluten medlem måste diskuteras på nästa stormöte. Antingen så upphävs då uteslutningen eller så bannlyses medlemmen permanent.

\section{Organisation}
\subsection{Verksamhetsåret}
Föreningens verksamhetsår sträcker sig mellan 1 juli till och med 30 juni nästföljande kalenderår.
\subsection{Stormöte}
Ett stormöte är föreningens högsta beslutande organ.
\subsubsection{Medlemmarnas rättigheter}
Alla medlemmar har närvaro-, yttrande-, och förslagsrätt vid stormöte. Endast studerandemedlemmar har automatiskt rösträtt.
Stormöten kan ge  tillfällig rösträtt till stödmedlemmar efter majoritetsbeslut.
\subsubsection{Kallelse}\label{sec:kallelse}
En kallelse till stormöte, samt en preliminär föredragningslista, anslås på föreningens hemsida minst fyra (4) veckor innan utsatt datum.
\subsubsection{Föredragningslista}\label{sec:dagordning}
Den slutgiltiga föredragningslistan måste anslås på föreningens hemsida tidigast två (2) veckor och senast en (1) vecka innan utsatt datum för stormötet. Efter detta får föredragningslistan ej ändras innan mötet. Varje föredragningslista måste innehålla minst följande punkter:
\begin{itemize}
	\item Val av mötesordförande
	\item Val av mötessekreterare
	\item Val av justeringsperson tillika rösträknare
	\item Fastställande av röstlängden
	\item Beslut om mötets stadgeenliga utlysande
	\item Motioner och propositioner
	\item Övriga frågor
\end{itemize}
\subsubsection{Motioner}
Motioner måste skickas in skriftligen till den sittande styrelsen senast två (2) veckor innan utsatt datum för ett stormöte. De blir sedan en del av den slutgiltiga föredragningslistan tillsammans med ett motionssvar från den sittande styrelsen.
\subsubsection{Beslut}
Beslut fattas med enkel majoritet. Röstning med fullmakt får ej förekomma. Medlemmar kan begära votering. Vid lika röstetal har mötesordföranden utslagsröst. För att föra in ett nytt ärende på föredragningslistan erfordras 75\% majoritet. Under punkten  ``övriga frågor'' får det ej behandlas frågor som gäller kostnader. Stormöte är beslutsmässigt.
\subsubsection{Val av funktionär}
Vid val av funktionär äger alla medlemmar rätt att nominera och kandidera för poster. Alla nominerade och kandiderande ska ges chansen att presentera sig själva och varför de vill besitta posten.
\subsubsection{Adjungeringar}\label{sec:adjungering}
Stormötet kan adjungera personer. Med adjungering avses närvaro-, yttrande- och förslagsrätt. Adjungering medför ej rätt att deltaga i beslut, ej heller medansvar för fattade beslut.

\subsubsection{Protokoll}
Stormöten måste protokollföras och protokollen förfärdigas i minst två exemplar, varav ett skall anslås på föreningens hemsida och ett skall arkiveras. Protokoll ska vara färdigställda inom två (2) veckor efter ett möte.
\subsubsection{Justering av protokoll}
Protokoll från stormöte skall justeras av mötesordföranden, mötessekreteraren och en av mötet utsedd justeringsperson.
\subsection{Vårmötet}
På våren varje verksamhetsår skall ett stormöte hållas. Detta kallas vårmötet. Vårmötmötets föredragningslista måste, förutom de som nämns i \ref{sec:dagordning}, minst lyfta punkterna:
\begin{itemize}
    \item Fastställande av föredragningslista
	\item Fastställandet av nästa verksamhetsårs medlemsavgift
	\item Val av nästföljande verksamhetsårs styrelse
	\item Val av nästföljande verksamhetsårs revisor
	\item Fastställande av nästkommande verksamhetsårs budget
	\item Nuvarande verksamhetsårs styrelses verksamhetsberättelse
	\item Nuvarande verksamhetsårs styrelses ekonomiska berättelse
	\item Revisorns granskning av nuvarande verksamhetsårs styrelses arbete
\end{itemize}
\subsubsection{Extra stormöte}
Vid behov kan ett extra stormöte sammankallas av styrelseledamot, revisorn eller en grupp av minst 50\% av aktiva studerandemedlemmarna. Vid yrkande om extra stormöte skall en kallelse fastslås inom två (2) veckor.
\subsection{Höstmötet}
På hösten varje verksamhetsår ska ett stormöte hållas. Detta kallas höstmötet. Höstmötets föredragningslista måste, förutom de som nämns i paragraf 3.2.3, minst lyfta punkterna
\begin{itemize}
    \item Fastställande av föredragningslista
	\item Föregående verksamhetsårs styrelses verksamhetsberättelse
	\item Föregående verksamhetsårs styrelses ekonomiska berättelse
	\item Revisorns granskning av föregående verksamhetsårs styrelses arbete
	\item Beslut om ansvarsfrihet av föregående verksamhetsårs styrelse
\end{itemize}
\subsection{Styrdokument}
Föreningens verksamhet regleras av dessa stadgar. För att ändra i stadgarna krävs 75\% majoritet på ett stormöte.
\subsubsection{Tolkningsfrågor}
Om tolkningsfrågor skulle uppstå i styrdokumenten gäller styrelsens mening, tills frågan avgjorts på stormöte. Efter avklarad tolkningsfråga skall formuleringen som gav upphov till situationen justeras enligt stormötets beslut.
\subsubsection{Ändringshistorik}
Ändringar som görs i stadgarna ska noteras i en ändringshistorik i stadgarna.
\subsection{Entledigande}
Då särskilda skäl föreligger kan styrelsen efter skriftlig ansökan från funktionär entlediga vederbörande samt tillförordna annan person att fullgöra den entledigades uppgifter till nästa stormöte, då val skall ske. Styrelsen äger ej rätt att entlediga:
\begin{itemize}
	\item Ordförande
	\item Kassör
	\item Revisor
\end{itemize}
\subsubsection{Entledigande vid stormöte}
Stormöte kan vid behov entlediga vilken funktionär som helst och utse en ersättare.

\section{Styrelsen}
Styrelsen handhar ledning av föreningens verksamhet i enlighet med syftet, se \ref{sec:syfte}, under verksamhetsåret. Styrelsen består minst av följande ordinarie ledamöter:
\begin{itemize}
	\item Ordförande, se \ref{sec:ordf}
	\item Kassör, se \ref{sec:cash}
	\item Verksamhetsledare, se \ref{sec:event}
\end{itemize}

\subsection{Rättigheter och skyldigheter}
Det åligger styrelsen att:
\begin{itemize}
	\item Ha roligt
	\item Besluta om den löpande verksamheten
	\item Bereda ärenden, vilka skall behandlas vid stormöten
	\item Upprätta förslag till föredragningslista för stormöten
	\item Inför stormöten ansvara för föreningens verksamhet
	\item Verkställa av stormöten fattade beslut
	\item Förbereda sina efterträdare inför deras verksamhetsår
\end{itemize}
\subsection{Styrelsemöten}
Styrelsemöten måste hållas minst en gång per termin under verksamhetsåret. Styrelsemöten är beslutsmässiga då minst hälften av ledamöterna är närvarande.
\subsubsection{Adjungeringar}
Styrelsemötena likt stormötena kan adjungera personer, se \ref{sec:adjungering}
\subsection{Protokoll}
Styrelsemöten måste protokollföras och protokollen ska anslås på föreningens
hemsida. Protokollen för ett verksamhetsår ska finnas tillgänliga tills dess att
det verksamhetsårets styrelse beviljats ansvarsfrihet. Protokoll måste justeras
av ordföranden och sekreteraren. Protokoll ska vara färdigställda inom två (2)
veckor efter ett möte.
\subsection{Firmateckning}
Föreningens firma, om sådan finns, tecknas av Ordföranden och Kassören var för sig.
\subsection{Ordförande}\label{sec:ordf}
Det åligger ordföranden att:
\begin{itemize}
	\item Representera föreningen och agera kontaktperson i officiella sammanhang
	\item Leda föreningens organisation
	\item Se till att föreningens verksamhet sker i enlighet med gällande styrdokument
	\item Hantera skötsel och uppdatering av föreningens styrdokument
	\item Handha och uppdatera medlemsregistret
\end{itemize}
\subsection{Kassör}\label{sec:cash}
Det åligger kassören att:
\begin{itemize}
	\item Vid sammanträden föra protokoll
	\item Upprätta budget för nästkommande verksamhetsår
	\item Sköta föreningens bokföring. Syftena med bokföringen, utan prioritetsordning, är följande:
\begin{itemize}
	\item Att möjliggöra kontroll av föreningens verksamhet, genom revisorn/revisorernas försorg
	\item Att underlätta för nästkommande års verksamhet
\end{itemize}
\item Föra kund- och leverantörsreskontra, betala fakturor i tid samt följa upp icke betalda fordringar.
\item Inventarie-/lagerföra föreningens tillgångar
\item Vid varje stormöte eller vid anmodan redovisa föreningens ekonomiska ställning
\item Ansvara för föreningens avtal och arkivering
\end{itemize}
\subsection{Verksamhetsledare}\label{sec:event}
Det åligger verksamhetsledaren att:
\begin{itemize}
\item Anordna aktiviteter i enlighet med syftet, se \ref{sec:syfte} Aktiviteter bör anordnas både exklusivt för föreningens medlemmar för att uppmuntra till medlemskap såväl som öppna för allmänheten
\item Informera föreningens medlemmar om dessa aktiviteter
\end{itemize}

\section{Revision}
En revisor skall väljas på vårmötet som ska granska föreningens verksamhet. Revisorn ska agera både sak- och sifferrevisor. Revisorn skall vara myndig och får ej vara jävig.
\subsection{Åligganden}
Revisorn skall före höstmötet avsluta sin granskning av föregående års verksamhet och över den företagna revisionen upprätta revisionsberättelse.
\subsection{Revisionsberättelse}
Revisionsberättelse skall innehålla yttrande i fråga om ansvarsfrihet för berörda 
funktionärer. 
\subsection{Handlingar}
Räkenskaper och övriga handlingar skall tillställas revisorn senast två(2) veckor före höstmötet.
\subsection{Avgång}
Om någon befattningshavare inom föreningen avgår, skall granskning av dennes 
förvaltning genast verkställas.
\subsection{Rättigheter}
Revisorn har rätt att närvara vid styrelsemötena. Revisorn skall ha insikt till föreningens tillgångar. Revisorn kan anmoda förtroendevalda att lämna ut information som behövs för en korrekt revision.

\end{document}
