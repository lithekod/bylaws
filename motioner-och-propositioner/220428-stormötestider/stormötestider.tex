\documentclass[12pt,a4paper]{article}
%% Setup
\usepackage[utf8]{inputenc}
\usepackage{amsmath}
\usepackage{graphicx}
\usepackage{amsfonts}
\usepackage{amssymb}
\usepackage{caption}
\usepackage{hyperref}
\usepackage{float}
\usepackage{tabularx}
\usepackage{fancyhdr}
\usepackage[swedish]{babel}  
\usepackage[left=2cm,right=2cm,top=2.5cm,bottom=2.5cm]{geometry}
\usepackage{lastpage}
\usepackage{titlesec}

\usepackage{draftwatermark}
\SetWatermarkScale{5}

\usepackage[normalem]{ulem}
\setlength\parindent{0pt}

\usepackage{framed}
\newenvironment{quotationb}%
{\begin{leftbar}}%
{\end{leftbar}}

\usepackage{listings}
\lstset{literate=%
{ö}{{\o}}1
{ä}{{\ae}}1
{æ}{{\ae}}1
{å}{{\aa}}1
{ø}{{\o}}1
{Æ}{{\AE}}1
{Å}{{\AA}}1
{Ø}{{\O}}1
}
\lstset{extendedchars=\true}
\lstset{inputencoding=ansinew}

\titleformat{\section}{\Large\bfseries}{}{0em}{}
\titleformat{\subsection}{\bfseries}{}{0em}{}

% Ändra gärna datumet här från \today till t.ex. "2022-02-29"
\newcommand{\thedate}{\today}

\begin{document}

\pagestyle{fancy}
\lhead{\includegraphics[height=4em]{lithekod.png}}
\rhead{Proposition\\Linköping \thedate\\\phantom{}}
\pagenumbering{gobble}

\hspace{20em}

% Ta bort den här innan stormötet.
\SetWatermarkText{Utkast}

% Motionens titel, t.ex. "Ändring av stadgarna för att efterleva GDPR"
\section{Förtydligande av hållpunkter inför stormötenj}

\subsection{Bakgrund}

% Här skriver du bakgrunden till motionen, t.ex. "med anledning av ..."

\subsection{Förslag}

% Beskriv vad du vill att föreningen ska göra. Till exempel "ändra i stadgarna
% paragraf xyz till följande: "blablabla".

\begin{enumerate}
    \item Ändra \S{}3.2.2 i stadgarna från 

    \begin{quotationb}
        En kallelse till stormöte, samt en preliminär föredragningslista, anslås
        på föreningens hemsida minst fyra (4) veckor innan utsatt datum.
    \end{quotationb}

    till

    \begin{quotationb}
        En kallelse till stormöte, samt en preliminär föredragningslista, anslås
        på föreningens hemsida \sout{minst} \underline{senast} fyra (4) veckor
        innan utsatt datum.
    \end{quotationb}

    \item Ändra första meningen i \S{}3.2.3 i stadgarna från

    \begin{quotationb}
        Den slutgiltiga föredragningslistan måste anslås på föreningens hemsida
        senast en (1) vecka innan utsatt datum för stormötet.
    \end{quotationb}

    till 

    \begin{quotationb}
        Den slutgiltiga föredragningslistan måste anslås på föreningens hemsida
        \underline{tidigast två (2) veckor och} senast en (1) vecka innan utsatt datum för stormötet.
    \end{quotationb}

\end{enumerate}

\subsection{Yrkande}

% Sammanfatta ditt yrkande. Var väldigt tydligt med vad du vill att föreningen ska göra. Exempel: "Jag
% yrkar att föreningen antar stadgeändringarna som presenteras ovan." Håll dig
% gärna till en mening

Styrelsen yrkar att stormötet antar ovan presenterade ändringar till stadgarna.

\noindent\rule{\linewidth}{1pt}

Styrelsen genom Gustav Sörnäs

\end{document}
