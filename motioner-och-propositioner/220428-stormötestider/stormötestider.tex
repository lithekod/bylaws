\documentclass[12pt,a4paper]{article}
%% Setup
\usepackage[utf8]{inputenc}
\usepackage{amsmath}
\usepackage{graphicx}
\usepackage{amsfonts}
\usepackage{amssymb}
\usepackage{caption}
\usepackage{hyperref}
\usepackage{float}
\usepackage{tabularx}
\usepackage{fancyhdr}
\usepackage[swedish]{babel}  
\usepackage[left=2cm,right=2cm,top=2.5cm,bottom=2.5cm]{geometry}
\usepackage{lastpage}
\usepackage{titlesec}

\usepackage{draftwatermark}
\SetWatermarkScale{5}

\usepackage[normalem]{ulem}
\setlength\parindent{0pt}
\setlength\parskip{0.5em}

\usepackage{framed}
\newenvironment{quotationb}%
{\begin{leftbar}}%
{\end{leftbar}}

\usepackage{listings}
\lstset{literate=%
{ö}{{\o}}1
{ä}{{\ae}}1
{æ}{{\ae}}1
{å}{{\aa}}1
{ø}{{\o}}1
{Æ}{{\AE}}1
{Å}{{\AA}}1
{Ø}{{\O}}1
}
\lstset{extendedchars=\true}
\lstset{inputencoding=ansinew}

\titleformat{\section}{\Large\bfseries}{}{0em}{}
\titleformat{\subsection}{\bfseries}{}{0em}{}

% Ändra gärna datumet här från \today till t.ex. "2022-02-29"
\newcommand{\thedate}{2022--04--28}

\begin{document}

\pagestyle{fancy}
\lhead{\includegraphics[height=4em]{lithekod.png}}
\rhead{Proposition\\Linköping \thedate\\\phantom{}}
\pagenumbering{gobble}

\hspace{20em}

% Ta bort den här innan stormötet.
\SetWatermarkText{}

% Motionens titel, t.ex. "Ändring av stadgarna för att efterleva GDPR"
\section{Förtydligande av hållpunkter inför stormöten}

\subsection{Bakgrund}

% Här skriver du bakgrunden till motionen, t.ex. "med anledning av ..."

I nuläget finns det en möjlig tolkning av stadgarna där ett stormöte annonseras
fyra veckor innan utsatt datum och sedan direkt fastställs utan att någon
möjlighet ges för inskickning av motioner. Motioner får enligt stadgarna skickas
in fram tills två veckor innan utsatt datum, men eftersom föredragningslistan
inte får ändras efter ett fastställande måste motionen istället lyftas som en ny
punkt på mötet med begränsningarna det innebär. Den här propositionen
förtydligar att ett fastställande endast får göras efter tidsperioden för
inskickandet av motioner gått ut, två veckor innan utsatt datum.

Syftet med propositionen är i slutändan \emph{inte} att förhindra en illa
menande styrelse från att förhindra att motioner skickas in, utan istället att
låta stadgarna verka vägledande för hur stormöten ska utannonseras och när.

\subsection{Förslag}

% Beskriv vad du vill att föreningen ska göra. Till exempel "ändra i stadgarna
% paragraf xyz till följande: "blablabla".

Ändra första meningen i \S{}3.2.3 i stadgarna från

\begin{quotationb}
    Den slutgiltiga föredragningslistan måste anslås på föreningens hemsida
    senast en (1) vecka innan utsatt datum för stormötet.
\end{quotationb}

till 

\begin{quotationb}
    Den slutgiltiga föredragningslistan måste anslås på föreningens hemsida
    \underline{tidigast två (2) veckor och} senast en (1) vecka innan utsatt datum för stormötet.
\end{quotationb}

\subsection{Yrkande}

% Sammanfatta ditt yrkande. Var väldigt tydligt med vad du vill att föreningen ska göra. Exempel: "Jag
% yrkar att föreningen antar stadgeändringarna som presenteras ovan." Håll dig
% gärna till en mening

Styrelsen yrkar att stormötet antar ovan presenterade ändringar till stadgarna.

\noindent\rule{\linewidth}{1pt}

Styrelsen genom Gustav Sörnäs

\end{document}
