\documentclass[12pt,a4paper]{article}
%% Setup
\usepackage[utf8]{inputenc}
\usepackage{amsmath}
\usepackage{graphicx}
\usepackage{amsfonts}
\usepackage{amssymb}
\usepackage{caption}
\usepackage{hyperref}
\usepackage{float}
\usepackage{tabularx}
\usepackage{fancyhdr}
\usepackage[swedish]{babel}  
\usepackage[left=2cm,right=2cm,top=2.5cm,bottom=2.5cm]{geometry}
\usepackage{lastpage}
\usepackage{titlesec}

\usepackage{draftwatermark}
\SetWatermarkScale{5}

\usepackage[normalem]{ulem}
\setlength\parindent{0pt}
\setlength\parskip{0.5em}

\usepackage{framed}
\newenvironment{quotationb}%
{\begin{leftbar}}%
{\end{leftbar}}

\usepackage{listings}
\lstset{literate=%
{ö}{{\o}}1
{ä}{{\ae}}1
{æ}{{\ae}}1
{å}{{\aa}}1
{ø}{{\o}}1
{Æ}{{\AE}}1
{Å}{{\AA}}1
{Ø}{{\O}}1
}
\lstset{extendedchars=\true}
\lstset{inputencoding=ansinew}

\titleformat{\section}{\Large\bfseries}{}{0em}{}
\titleformat{\subsection}{\bfseries}{}{0em}{}

% Ändra gärna datumet här från \today till t.ex. "2022-02-29"
\newcommand{\thedate}{\today}

\begin{document}

\pagestyle{fancy}
\lhead{\includegraphics[height=4em]{lithekod.png}}
\rhead{Proposition\\Linköping \thedate\\\phantom{}}
\pagenumbering{gobble}

\hspace{20em}

% TODO: Ta bort den här innan stormötet.
\SetWatermarkText{Utkast}

% Motionens titel, t.ex. "Ändring av stadgarna för att efterleva GDPR"
\section{Införande av ändringshistorik i stadgarna}

\subsection{Bakgrund}

I nuläget finns ingen ändringshistorik över stadgarna. Med en ändringhistorik
kan efterföljande styrelser och medlemmar se vilka ändringar som skett, när dom
skett och hur diskussionen såg ut när ändringen skulle införas.

Ett förslag på hur en ändringshistorik skulle se ut bifogas.

\subsection{Förslag}

% Beskriv vad du vill att föreningen ska göra. Till exempel "ändra i stadgarna
% paragraf xyz till följande: "blablabla".

Inför en ny underrubrik \S3.5.2 ''Ändringshistorik'' i stadgarna som lyder

\begin{quotationb}

Ändringar som görs i stadgarna ska noteras i en ändringshistorik i stadgarna.

\end{quotationb}

\subsection{Yrkande}

% Sammanfatta ditt yrkande. Var väldigt tydligt med vad du vill att föreningen ska göra. Exempel: "Jag
% yrkar att föreningen antar stadgeändringarna som presenteras ovan." Håll dig
% gärna till en mening

Styrelsen yrkar att stormötet antar ovan presenterade tillägg till stadgarna.

\noindent\rule{\linewidth}{1pt}

Styrelsen genom Gustav Sörnäs

\end{document}
